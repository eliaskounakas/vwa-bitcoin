\section{Asymmetrische Kryptographie}
Die Kryptographie ist ein Zweig der Mathematik, welcher vor allem in der Cyber-Security Branche verwendet wird. Mit Kryptographie
kann man beispielsweise beweise, dass man ein Geheimniss kennt, ohne das Geheimniss zu verraten. Im Falle von Bitcoin wird es 
genutzt, um den Zugriff auf eine Bitcoin-Addresse zu beweisen. Erreicht wird dies mit dem Public-Key-Verfahren/Asymmetrische
Verschlüsselung. Hierbei wird ein Schlüsselpaar generiert, welches aus dem Public Key und dem Private Key besteht. Der Public Key
darf von anderen gesehen werden, der Private Key jedoch nicht. Generiert werden diese Schlüssel mit einem Algorithmus namens
ECDSA (Elliptic Curve Digital Signature Algorithm).

Transaktionen werden mit einer digitalen Signatur abgeschlossen, für welche der Private Key benötigt wird. Das heißt, dass jeder
mit Zugriff auf den Private Key Transaktionen abschließen kann. Den Private Key kann man sich also wie ein Passwort für ein 
Bankkonto vorstellen. Bitcoin Addressen werden meist aus dem Public Key generiert, indem dieser gehashed wird. Der Public Key
kann zwar von anderen gesehen werden, da es keine Sicherheitsrisiken kreiert, wird jedoch trotzdem gehashed. Grund dafür ist,
dass Speicherplatz in Bitcoin eine sehr große Rolle spielt. Public Keys haben nämlich mehr Bits als das Ergebnis einer
Hashfunktion.
