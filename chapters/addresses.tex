\section{Bitcoin-Adressen generieren}
In Bitcoin werden Adressen verwendet, um Transaktionen zu empfangen und zu senden. Eine Bitcoin-Adresse ist eine Zeichenfolge, 
die aus einer Reihe von Zahlen und Buchstaben besteht und einem öffentlichen Schlüssel entspricht.

Um eine Bitcoin-Adresse zu generieren, wird der öffentliche Schlüssel mithilfe einer kryptographischen Hashfunktion verarbeitet.
Die Hashfunktion nimmt den öffentlichen Schlüssel als Eingabe und gibt einen Hash als Ausgabe. Der öffentliche Schlüssel selbst
wird nicht als Adresse genutzt, weil der Hash weniger Speicherplatz benötigt. Der Hash wird dann mithilfe einer weiteren 
Mathematikfunktion, der Base58Check-Codierung, in eine lesbare Adresse umgewandelt.

\section{Base58Check-Codierung}
Um große Zahlen mit wenigen Zeichen darzustellen, nutzen Computersysteme häufig alphanumerische Zahlensysteme mit einer Basis,
die größer als 10 ist. Das Dezimalsystem nutzt die 10 Ziffern 0-9, während das Hexadezimalsystem zusätzlich die Buchstaben A-F
nutzt, wodurch es 16 Ziffern hat. Deshalb sind Zahlen, welche hexadezimal dargestellt werden, kürzer. Base64 ist ein weiteres
Zahlensystem, das noch viel kompakter ist. Es nutzt die 26 Buchstaben des Alphabets doppelt, nämlich in Klein- und Großschreibung.
Zusätzlich nutzt Base64 die 10 Ziffern 0-9 und die Symbole + sowie /.

Base58 hingegen ist eine Ableitung von Base64, welche ähnlich aussehende Symbole vermeidet, da sie in manchen Schriftarten
identisch aussehen können. Diese Symbole sind die Zahl null, der Buchstabe O (großes o), l (kleines L), I (großes i), das 
Zeichen + und /. Das vollständige Alphabet von Base58 ist: \[123456789ABCDEFGHJKLMNPQRSTUVWXYZabcdefghijkmnopqrstuvwxyz\]

Base58Check wiederum ist ein Codierungsformat von Base58, welches oft in Bitcoin genutzt wird und einen eingebauten
Fehlerüberprüfungs-Code besitzt. Die Prüfsumme wird vom Hash der codierten Daten abgeleitet und kann deswegen genutzt werden, 
um Tippfehler und andere Fehler zu erkennen. Wenn die Prüfsumme also nicht mit den codierten Daten übereinstimmt, gilt die
eingegebene Bitcoin-Adresse als ungültig. Dies verhindert den Versand an falsche Bitcoin-Adressen und dadurch den Verlust
von Geld. (Antonopoulos, 67)

Die meisten Daten in Bitcoin werden Base58Check-codiert, was es einfacher macht, mit den Daten zu arbeiten. Zudem wird ein Präfix
bei jeder Base58Check-Codierung genutzt, welche den Datentyp angibt, um welchen es sich handelt. Das ermöglicht dem Nutzer, schnell
zu erkennen, welche Art von Daten dargestellt werden. Beispielsweise haben Bitcoin-Adressen in Base58Check den Präfix 0x00, welcher
umgewandelt auf Base58 1 ergibt. Private Schlüssel wiederum haben den Präfix 0x80 in Base58Check bzw. 5 in Base58.

\section{Private Schlüssel}
Im Bitcoin-Netzwerk ist der private Schlüssel eine geheime Zahl, die genutzt wird, um Bitcoin-Ausgaben ausgehend von einer 
bestimmten Bitcoin-Adresse zu authorisieren. Der private Schlüssel ist eine kritische Komponente der kryptographischen Sicherheit,
auf welche das Bitcoin-Netzwertk aufbaut. Genutzt wird der private Schlüssel, um Authentizität bei Transaktionen zu beweisen und
versichert dabei, dass die Transaktionen nicht verändert oder gefälscht sind.

Meist sind private Schlüssel in einer sogenannten Wallet, eine Applikation, welche dem Nutzer ermöglicht, Transaktionen zu tätigen,
gespeichert. Jede Wallet hat einen einzigartigen privaten Schlüssel, welche für Unterschriften bei Transaktionen genutzt wird und
den Besitz der jeweiligen Bitcoins beweist. Es ist essentiell, dass der private Schlüssel geschützt ist und mit niemanden geteilt
wird, da jeder mit Zugriff auf den privaten Schlüssel Bitcoins ausgeben kann.

Private Schlüssel werden als Abfolge von Zahlen und Buchstaben dargestellt. Genauer gesagt werden private Schlüssel durch ein
bestimmtes Format wie das Base64 Zahlensystem codiert. Folgende Hexadezimalzahl zeigt, wie ein privater Schlüssel im 
Normalfall aussieht: 8a708d03d461a5c5839b53da4c40912ca094cc0edee8574a62d6895d033db7ea. Diese Zahl entspricht einem Dezimalwert 
von ungefähr \( 6.2 * 10^{70} \). Generiert werden private Schlüssel mit verschiedenen Methoden wie beispielsweise ein 
Zufallszahlengenerator. Gespeichert werden private Schlüssel im Normalfall auf einem Computer oder Smartphone in einer sicheren 
Datei.

\subsection{Private Schlüssel generieren}
Prinzipiell sind private Schlüssel eine zufällige Zahl zwischen 1 und \( 2^{256} \) bzw. \( 1,15 * 10^{77} \). Um diese enorme Zahl
zu relativisieren: die geschätzte Anzahl der Atome im sichtbaren Universum beträgt \( 10^{80} \). (Antonopoulos, 59) Die Chance,
zwei mal denselben privaten Schlüssel zu generieren ist also extrem gering.
