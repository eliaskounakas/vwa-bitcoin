\subsection{Skalierbarkeit}
Da Bitcoin ein selbstregulierendes System ist, welches durch das ''Mining'' von Blöcken alle 10 Minuten funktioniert, ist die
höchste Anzahl von Transaktionen prinzipiell die maximale Größe eines Blocks dividiert durch das Mining-Intervall. Bei einem
Intervall von 10 Minuten, um Transaktionen zu bestätigen, schafft es Bitcoin auf einem Transaktionsdurchsatz von 7 Transaktionen 
pro Sekunde zu kommen. Um dies in Vergleich zu stellen, bei einer herkömmlichen Zahlungsmethode wie zum Beispiel VISA werden
Transaktion innherhalb von Sekunden bestätigt. Zudem liegt der Transaktionsdurchsatz von VISA bei über 2000 Transaktionen pro
Sekunde. Offensichtlicherweise existiert eine riesige Lücke zwischen dem Stand, auf welchem Bitcoin sich befinden und
herkömmlichen Zahlungsmethoden. (Vgl. Croman et al., 2016, S. 1) Weiters ist die Zeit, die eine neue Full Node benötigt, um die gesamte 
Blockchain herunterzuladen enorm. Die derzeitige Größe der Bitcoin-Blockchain beträgt 425 Gigabyte, während die
durchschnittliche Internetgeschwindigkeit bei etwa 30Mbps liegt. Folgend ist die benötigte Zeit \(\frac{425000}{3.75} = 113333\)
Sekunden beziehungsweise 31.5 Stunden. Für die meisten Nutzer ist es daher undenkbar, eine Full Node zu betreiben, obwohl Bitcoin
nur dadurch sicher bleiben kann. 

Um solche Probleme zu lösen, müssen sogenannte ''Forks'' das System von Grund auf ändern.

Eine Art von Fork wird als Hard Fork bezeichnet, da das Netzwerk nach dem Fork nicht wieder auf eine einzige Blockchain 
konvergiert. Stattdessen entwickeln sich die beiden Blockchains unabhängig voneinander weiter. Hard Forks treten auf, wenn ein 
Teil des Netzwerkes unter einem anderen Satz von Konsensregeln arbeitet als der Rest des Netzwerkes. Dies kann aufgrund eines
Fehlers oder aufgrund einer bewussten Änderung in der Implementierung der Konsensregeln auftreten. Hard Forks können verwendet
werden, um die Konsensregeln zu ändern, erfordern jedoch eine Koordination zwischen allen Teilnehmern im System. Alle Nodes,
die nicht auf die neuen Konsensregeln aktualisieren, können nicht am Konsensmechanismus teilnehmen und werden zum Zeitpunkt des
Hard Forks auf eine separate Kette gezwungen. Eine Änderung, die durch einen Hard Fork eingeführt wird, kann also als nicht
''vorwärtskompatibel'' betrachtet werden, da nicht aktualisierte Systeme die neuen Konsensregeln nicht mehr verarbeiten können.
(Vgl. Antonopoulos, 2017, S. 257)

Nicht alle Änderungen der Konsensregeln führen zu einem Hard Fork. Nur Konsensänderung, die nicht vorwärtskompatibel sind,
verursachen einen Fork. Wenn die Änderung so implementiert wird, dass ein unveränderter Client die Transaktion oder den
Block weiterhin als gültig gemäß den vorherigen Regeln betrachtet, kann die Änderung ohne Fork erfolgen. Der Begriff
''Soft Fork'' wurde eingeführt, um diese Upgrade-Methode von einem ''Hard Fork'' zu unterscheiden. In der Praxis ist ein Soft
Fork überhaupt kein Fork. Ein Soft Fork ist eine vorwärtskompatible Änderung der Konsensregeln, die es nicht aktualisieren
Clients weiterhin ermöglicht, im Konsens mit den neuen Regeln zu arbeiten. Soft Forks können auf verschiedene Arten implementiert
werden - der Begriff definiert keine spezifische Methode, sondern eine Reihe von Methoden, die alle eines gemeinsam haben: Sie
erfordern nicht, dass alle Nodes aktualisiert werden oder zwingen nicht aktualisierte Nodes aus dem Konsens. (Vgl. Antonopoulos,
2017, S. 261)

\subsection{Verluste}
Kritiker betonen oft die fehlende physikalische Form von Bitcoin, und das zurecht. Bitcoins können nur in der digitalen Welt
existieren und man ist dadurch auf Wallets angewiesen. Wenn man in irgendeiner Art und Weise Zugriff auf diese Wallet verliert,
sind die darauf enthaltenen Schlüssel (und im Zuge dessen auch die Bitcoins) für immer verloren. Das kann bei digitalen Wallets 
sehr schnell gehen, wenn beispielsweise
die Festplatte, auf welcher die Wallet betrieben wird, kaputt geht. Oder wenn man bei einem Online-Anbieter sein Passwort
vergisst. Anders als bei Banken, bei welchen man trotz verlorener Kreditkarte Zugriff auf sein Konto bekommen kann, gibt es 
bei Bitcoin keine zentrale Einheit, die solche Problematiken löst.

Bitcoin ist derzeit so konzipiert, dass es in Einheiten names Satoshis unterteilbar ist, die 0.00000001 eines Bitcoins
entsprechen. Die Tatsache, dass Bitcoins so teilbar sind, soll bedeuten, dass aufrund der asymptotischen Grenze von 21
Millionen Bitcoins * 100.000.000 Satoshis pro Bitcoin insgesamt 2.100.000.000.000.000 (2,1 Billiarden) Satoshis möglich sind.
Dadurch kann Bitcoin eine lebensfähige globale Währung sein, die praktisch jede Art von Expansion im Maßstab von Nationen 
unterstützen kann. (Vgl. Hanley, 2013, S. 7)
Hier tritt das Problem der Verluste in einer neuen Form auf: Wenn Wallets verloren gehen, dann sind auch die 
darauf enthaltenen Bitcoins für immer aus dem Bitcoin-Netzwerk verloren. Die einzige Methode, mit welcher man die Bitcoins
zurückholen könnte, ist alle Private-Schlüssel-Kombinationen auszuprobieren, bis man auf eine Wallet stoßt, auf welcher Bitcoins
enthalten sind. Diese Methode ist jedoch extrem rechenintensiv und praktisch unmöglich. Da im Laufe der Zeit immer mehr Wallets
verloren gehen, wird die Anzahl der Bitcoins auch immer geringer.