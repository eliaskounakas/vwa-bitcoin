\section{Elliptische Kurve}
Elliptische Kurven sind die Menge aller Punkte in den reellen Zahlen, die die Gleichung $ 4y^2 = x^3 + ax + b $ erfüllen, wobei
$ 4a^3 + 27b^2 \neq 0 $, um Singularitäten auszuschließen. Diese Form der elliptischen Kurve nennt man die Weierstraß-Normalform.
Außerdem wird ein Punkt im Unendlichen benötigt, welcher Teil der Kurve ist. Für den Zweck dieser Arbeit wird der Punkt im 
Unendlichen als $\infty$ bezeichnet. Um den Punkt der Unendlichkeit explizit zu definieren, lautet die vollständige Definition (vgl andrea
corbellini): \[ \{ (x,y) \in \mathbb{R}^2 | y^2 = x^3 + ax + b, 4a^3 + 27b^2 \neq 0 \} \cup \{ \infty \} \]

\subsection{Gruppenoperationen}