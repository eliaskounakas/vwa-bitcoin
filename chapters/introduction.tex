\section{Was ist Bitcoin?}
Häufig wird Bitcoin einzig und allein mit der Kryptowährung selbst assoziiert. Der Begriff Bitcoin umfasst jedoch alle 
Konzepte und Technologien, die das digitale Zahlungssystem ermöglichen. Nutzer können über das sogenannte Bitcoin-Netzwerk 
Zahlungen propagieren und verifizieren. Mit der Kryptowährung Bitcoin kann man Käufe tätigen, Geld an Verwandte oder
Organisationen versenden und Waren verkaufen. Der große Unterschied zwischen Bitcoin und herkömmlicher Währung ist, dass es 
kein handgreifliches Bitcoin gibt. Die Menge an Bitcoin, die man besitzt, wird durch die Zusammenfassung aller erhaltenen 
Transaktionen erlangt. Dieser Wert kann bis zu acht Dezimalstellen besitzen, da die kleinste Einheit von Bitcoin ein Satoshi
ist. 100 Millionen Satoshis entsprechen einem bitcoin.

Transaktionen werden mithilfe von assymetrischer Verschlüsselung abgeschlossen. Diese setzt sich aus dem Private Key und Public
Key zusammen. Mit dem Public Key wird die Bitcoin-Addresse des Nutzers hergestellt, welche benötigt wird, um Zahlungen zu
empfangen. Ähnlich wie der PIN oder das Passwort bei einem herkömmlichen Bankkonto lassen sich Zahlungen mit dem Private Key
im Bitcoin-Netzwerk tätigen.

In Bitcoin gibt es keine zentrale Macht oder Bank, die neues Geld produziert. Stattdessen werden neue Bitcoins mit einem Prozess
names Mining gewonnen, welcher die Lösung zu einem mathematischen Problem sucht und gleichzeitig neue Transaktionen verifiziert.
Alle 10 Minuten wird eine neue Lösung gefunden und alle Miner starten wieder von vorne. Das bedeutet, dass jeder Nutzer auch ein
Miner sein kann und Transaktionen maximal 10 Minuten brauchen, um verifiziert zu werden.


\section{Geschichte von Bitcoin}
2008 veröffentlichte jemand unter dem Alias Satoshi Nakamoto "Bitcoin: A Peer-to-Peer Electronic Cash System". Mit dieser Arbeit
erfand Nakamoto die erste Kryptowährung, Bitcoin. Er kombinierte frühere Erfindungen mit seinen Ideen und erfand somit das erste
komplett dezentralisierte, elektronische Zahlungssystem. Die Schlüsselrolle spielte die Implementierung von einem verteilten
Aufwand der Prozessorkraft, wodurch alle 10 Minuten demokratisch für die korrekte Folge von Aktionen abgestimmt werden kann.

In dem Jahr 2011, als Nakamoto sich von seinem Projekt abwandte und es nicht länger optimisierte, stieg Bitcoin mit einem
Aufschwung von 3000\% in nur drei Monaten von 1USD auf 32USD. Explodiert ist der Wert jedoch erst 2018, als Bitcoin einen
Höchstwert von ~19,000USD erreichte. Im vierten Quartal des Jahres 2021 hat Bitcoin seinen bislang höchsten Wert (Stand: 24/08/2022)
von ~67,000USD erreicht.