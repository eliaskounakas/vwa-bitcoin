Bitcoin-Transaktionen sind Aufzeichnungen über Übertragungen von Bitcoins von einem Benutzer an einen anderen. Jede
Bitcoin-Transaktion besteht aus mindestens einer Eingabe- und einer Ausgabeadresse. Die Eingabe-Adresse stellt die Bitcoins
bereit, die übertragen werden sollen, und die Ausgabe-Adressen empfangen die übertragenen Bitcoins. Eine Bitcoin-Transaktion
kann auch eine Gebühr beinhalten, die an die Miner gezahlt wird, die die Transaktion bestätigen und in die Blockchain aufnehmen.
Es ist wichtig, zu beachten, dass Bitcoin-Transaktionen irreversibel sind, d.h. sie können nicht rückgängig gemacht werden,
nachdem sie einmal bestätigt wurden. Daher sollte man sorgfältig darauf achten, an die richtige Adresse zu senden, bevor man eine
Bitcoin-Transaktion durchführt.

\section{Wallets}
Eine Wallet besteht aus Software, die private und öffentliche Schlüssel enthält und die Blockchain nutzt, um Währungen zu senden
und zu empfangen. Die Währung in diesen Wallets wird in Form von Coins, wie beispielsweise Bitcoin, hinzugefügt. Um Coins oder
Währung empfangen oder senden zu können, muss eine Wallet erstellt werden. Die Währung selbst ist nicht an einem Ort in der Wallet
gespeichert, sondern befindet sich in der Form von Transaktionsaufzeichnungen in der Blockchain. (Vgl. Yadav, Kuri und Goar,
2020, S. 4) Demnach enthält die Wallet lediglich die Informationen (Schlüssel), um Bitcoins auszugeben, nicht aber die Währung
selbst.


\section{Eingaben und Ausgaben}
Die Blockchain ist ein global vereinbartes Register aller jemals durchgeführten Transaktionen und wird ständig als eine einzige 
verknüpfte Liste erweitert. Jede Transaktion in der Blockchain, zum Beispiel Alice, die Bob 10 Bitcoins zahlt, hat einen oder
mehrere Transaktionausgänge (TXO), die als Summen aus verwendbarer Bitcoins dienen. Hierbei muss beachtet werden, dass Bitcoin
Summen nicht in kleinere Summen aufgeteilt werden können, d.h., dass es im Normalfall einen Überschuss an Bitcoins gibt, der an 
den Sender zurückgezahlt wird, ähnlich wie beim Wechselgeld im Falle normaler Währung. Sobald Bob die TXO beziehungsweise 
die Summen seiner verwendbaren Bitcoins erhält, bezeichnet man sie als unausgegebene Transaktionausgänge (UTXO). Sie bleiben
UTXOs, bis der Besitzer (in diesem Beispiel Bob) sie einlöst, um jemand anderen zu bezahlen (in diesem Fall werden sie wieder
als ausgegebene TXOs bezeichnet). (Vgl. Konrad und Pinto, 2015, S. 1)

