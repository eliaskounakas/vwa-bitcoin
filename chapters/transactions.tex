Bitcoin-Transaktionen sind Aufzeichnungen über Übertragungen von Bitcoins von einem Benutzer an einen anderen. Jede
Bitcoin-Transaktion besteht aus mindestens einer Eingabe- und einer Ausgabeadresse. Die Eingabe-Adresse stellt die Bitcoins
bereit, die übertragen werden sollen, und die Ausgabe-Adressen empfangen die übertragenen Bitcoins. Eine Bitcoin-Transaktion
kann auch eine Gebühr beinhalten, die an die Miner gezahlt wird, die die Transaktion bestätigen und in die Blockchain aufnehmen.
Es ist wichtig, zu beachten, dass Bitcoin-Transaktionen irreversibel sind, d.h. sie können nicht rückgängig gemacht werden,
nachdem sie einmal bestätigt wurden. Daher sollte man sorgfältig darauf achten, an die richtige Adresse zu senden, bevor man eine
Bitcoin-Transaktion durchführt.

\section{Wallets}
Eine Wallet besteht aus Software, die private und öffentliche Schlüssel enthält und die Blockchain nutzt, um Währung zu senden
und zu empfangen. Die Währung in diesen Wallets wird in Form von Coins, wie beispielsweise Bitcoin, hinzugefügt. Um Coins oder
Währung empfangen oder senden zu können, muss eine Wallet erstellt werden. Die Währung selbst ist nicht an einem Ort in der Wallet
gespeichert, sondern befindet sich in der Form von Transaktionsaufzeichnungen in der Blockchain. (Vgl. Yadav, Kuri und Goar,
2020, S. 4) Demnach enthält die Wallet lediglich die Informationen (Schlüssel), um Währung auszugeben, nicht aber die Währung
selbst.

Es gibt zwei primäre Arten von Wallets, welche sich darin unterscheiden, ob die enthaltenen Schlüssel zusammenhängend sind oder
nicht. Die erste Art einer Wallet ist die nichtdeterministische Wallet, bei welcher jeder Schlüssel unabhängig von einer 
zufälligen Zahl generiert wird. Die Schlüssel in dieser Wallet sind NICHT zusammenhängend. Die zweite Art einer Wallet ist die
deterministische Wallet, bei welcher alle Schlüssel Derivate von einem Masterschlüssel, welcher auch als Samen bezeichnet wird.
Alle Schlüssel in dieser Art von Wallet sind zusammenhängend und können neu generiert werden, solange man den originalen Samen
beziehungsweise den Masterschlüssel hat. Es gibt mehrere Methoden zur Derivation von Schlüsseln, welche in deterministischen
Wallets genutzt werden. Die am häufigsten genutzte Derivationsmethode nutzt eine baumähnliche Struktur und wird als 
hierarchische deterministische (HD) Wallet bezeichnet. (Vgl. Antonopoulos, 2017, S. 93f)

\subsection{Deterministische Wallets}
Deterministische Wallets beinhalten private Schlüssel, welche alle von einem gemeinsamen Samen abstammen. Die privaten Schlüssel
werden vom Samen durch Einwegfunktionen beziehungsweise Hashfunktionken berechnet. Der Samen ist eine zufällig generierte Zahl,
welche mit anderen Daten wie zum Beispiel einer Index Nummer kombiniert wird, um dann mittels einer Hashfunktion einen privaten
Schlüssel generiert. Im Gegensatz zu der nichtdeterministischen Wallet benötigt man bei der deterministischen Wallet einzig und 
allein den Samen, um auf alle Schlüssel zugreifen zu können, da diese alle vom Samen abstammen. Aus diesem Grunde benötigt man
nur ein einziges Backup, welches bei der Erstellung der Wallet gespeichert wird. Vorteilhaft ist dabei auch die Mobilität der
Wallet: Man muss einzig und allein den Samen auf die neue Wallet-Umgebung transferieren, um wieder Zugriff auf alle Schlüssel
zu haben. (Vgl. Antonopoulos, 2017, S. 95)

Im BIP-32 standard wurde die häufigste Art der deterministischen Wallets implementiert: die hierarchische deterministische Wallet.
HD Wallets enthalten Schlüssel, die in einer Baumstruktur angeordnet sind. Ein Schlüssel kann Töchterschlüssel generieren, von 
welchen wiederum Enkelschlüssel generiert werden können, und so weiter, bis in die Unendlichkeit. Abgesehen von der Mobilität
hat die HD Wallet noch zwei weitere Vorteile gegenüber nichtdeterministischen Wallets. Erstens können bestimmte Äste von
Schlüsseln genutzt werden, um ihnen organisatorische Bedeutung zu verleihen. Beispielsweise kann ein Ast von Schlüsseln genutzt
werden, um Zahlungen zu senden und ein andere Ast von Schlüsseln erhält dann das Wechselgeld. Zweitens können Nutzer von HD
Wallets eine Reihe von öffentlichen Schlüsseln generieren, ohne Zugriff auf die zugehörigen privaten Schlüssel zu haben. Dadurch 
können die Schlüssel auf unsicheren Servern genutzt. (Vgl. Antonopoulos, 2017, S. 96)


\subsection{Nichtdeterministische Wallets}

\section{Eingaben und Ausgaben}
Die Blockchain ist ein global vereinbartes Register aller jemals durchgeführten Transaktionen und wird ständig als eine einzige 
verknüpfte Liste erweitert. Jede Transaktion in der Blockchain, zum Beispiel Alice, die Bob 10 Bitcoins zahlt, hat einen oder
mehrere Transaktionausgänge (TXO), die als Summen aus verwendbarer Bitcoins dienen. Hierbei muss beachtet werden, dass Bitcoin
Summen nicht in kleinere Summen aufgeteilt werden können, d.h., dass es im Normalfall einen Überschuss an Bitcoins gibt, der an 
den Sender zurückgezahlt wird, ähnlich wie beim Wechselgeld im Falle normaler Währung. Sobald Bob die TXO beziehungsweise 
die Summen seiner verwendbaren Bitcoins erhält, bezeichnet man sie als unausgegebene Transaktionausgänge (UTXO). Sie bleiben
UTXOs, bis der Besitzer (in diesem Beispiel Bob) sie einlöst, um jemand anderen zu bezahlen (in diesem Fall werden sie wieder
als ausgegebene TXOs bezeichnet). (Vgl. Konrad und Pinto, 2015, S. 1)

