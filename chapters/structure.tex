\section{Peer-to-Peer-Netzwerk}
Im Bitcoin-Netzwerk gibt es keine zentrale Macht. Alle verbundenen Computer arbeiten miteinander als Peers. Peer-to-Peer
bzw. P2P heißt, dass alle teilnehmenden Computer die gleichen Rechte und Hindernisse haben. Die einzelnen verbundenen Computer
nennt man Nodes. Jede Node kommuniziert rund um die Uhr mit ihren benachbarten Nodes. Nodes können Informationen abfragen, aber 
auch weiterleiten. Das Bitcoin-Netzwerk ist demnach die Gesamtheit aller Nodes.

Was passiert aber, wenn eine Node für bösartige Zwecke falsche Informationen an andere Nodes verschickt? Dafür hat man ein
Vertrauens-System im Bitcoin-Netzwerk implementiert. Bevor eine Node empfangene Informationen abspeichert, prüft sie diese
auf Korrektheit in Bezug auf den derzeitigen Standard. Dieser Standard wird im BIP (Bitcoin Improvement Proposal) festgelegt,
welcher periodisch immer neue Richtlinien festlegt. Natürlich hat das BIP keine Macht über die Nodes, diese können nämlich
selber entscheiden, ob sie nun nach den neuen Richtlinien arbeiten wollen oder nicht. Jedoch übernehmen die meisten Nodes mit
einem kurzen Update den neuen Standard, da dieser lediglich für das Interesse der Teilnehmer entwickelt wird.

Bitcoin Nodes sind kaum leistungshungrig und benötigen wenig Energie. Deswegen ist es für so gut wie jede Person möglich,
selber zu Hause eine Bitcoin Node zum Laufen zu bringen. Die Zwecke für diese Node könnnen aber von Person zu Person variieren.
Zum einen gibt es Leute, die mit ihrer Node nur mithören wollen, zum anderen aber auch diejenigen, die durch Mining Geld
verdienen wollen. Aus diesem Grunde gibt es verschiedene Implementierungen und Rollen von Nodes.

\subsection{Full Node}

\subsection{Mining Node}
\subsection{SPV Client}

\section{Die Blockchain}
\subsection{Struktur eines Blocks}
\subsection{Genesis-Block}
