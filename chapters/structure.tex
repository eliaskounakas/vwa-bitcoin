\section{Peer-to-Peer-Netzwerk}
Im Bitcoin-Netzwerk gibt es keine zentrale Macht. Alle verbundenen Computer arbeiten miteinander als Peers. Peer-to-Peer
bzw. P2P heißt, dass alle teilnehmenden Computer die gleichen Rechte und Hindernisse haben. Die einzelnen verbundenen Computer
nennt man Nodes. Jede Node kommuniziert rund um die Uhr mit ihren benachbarten Nodes. Nodes können Informationen abfragen, 
weiterleiten und verifizieren. Das Bitcoin-Netzwerk ist demnach die Gesamtheit aller Nodes.

Was passiert aber, wenn eine Node für bösartige Zwecke falsche Informationen an andere Nodes verschickt? Dafür hat man ein
Vertrauens-System im Bitcoin-Netzwerk implementiert. Bevor eine Node empfangene Informationen abspeichert, prüft sie diese
auf Korrektheit in Bezug auf den derzeitigen Standard. Dieser Standard wird im BIP (Bitcoin Improvement Proposal) festgelegt,
welcher periodisch immer neue Richtlinien festlegt. Natürlich hat das BIP keine Macht über die Nodes, diese können nämlich
selber entscheiden, ob sie nun nach den neuen Richtlinien arbeiten wollen oder nicht. Jedoch übernehmen die meisten Nodes mit
einem kurzen Update den neuen Standard, da dieser lediglich für das Interesse der Teilnehmer entwickelt wird.

Bitcoin Nodes sind kaum leistungshungrig und benötigen wenig Energie. Deswegen ist es für so gut wie jede Person möglich,
selber zu Hause eine Bitcoin Node zum Laufen zu bringen. Die Zwecke für diese Node könnnen aber von Person zu Person variieren.
Zum einen gibt es Leute, die mit ihrer Node nur mithören wollen, zum anderen aber auch diejenigen, die durch Mining Geld
verdienen wollen. Aus diesem Grunde gibt es verschiedene Implementierungen und Rollen von Nodes.

\subsection{Full Node}
Eine Full Node benötigt von allen Nodetypen am meisten Speicherplatz, ist dafür aber die mächtigste und fasst alle Funktionen in 
einer Node zusammen. Der distinkte Unterschied von der Full Node ist, dass diese die komplette Blockchain herunterlädt und
verifiziert. Der benötigte Speicherplatz der Blockchain liegt bei 423GB (Stand: 25/8/2022) und steigt pro Monat um etwa 3GB an.

\subsection{Mining Node}
Damit die Blockchain erweitert wird und alle Transaktionen verfiziert werden könnnen, müssen sogenannte Miner alle 10 Minuten
eine Lösung zu einem mathematischen Problem finden, indem sie mit der Hashfunktion SHA256 einen bestimmten Wert suchen. Das muss
vor allem schnell und präzise geschehen, damit die Chance, die Lösung als Erstes zu finden, maximiert wird. Mining Nodes 
profitieren vor allem von modernen GPUs, welche hunderte Millionen Hashes pro Sekunde berechnen.

\subsection{SPV Client}
Die wahrscheinlich beste Node für den Otto Normalverbraucher ist der SPV Client. Dieser installiert lediglich eine Wallet und
stellt eine Verbindung zum Bitcoin-Netzwerk her. Er ist einzig und allein dafür ausgestattet, Transaktionen auf das eigene Konto
zu empfangen und Gled zu versenden. 

\section{Die Blockchain}
\subsection{Struktur eines Blocks}
\subsection{Genesis-Block}
