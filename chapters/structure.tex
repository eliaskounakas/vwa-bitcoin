\section{Peer-to-Peer-Netzwerk}
Im Bitcoin-Netzwerk gibt es keine zentrale Macht. Alle verbundenen Computer arbeiten miteinander als Peers. Peer-to-Peer
bzw. P2P heißt, dass alle teilnehmenden Computer die gleichen Rechte und Hindernisse haben. Die einzelnen verbundenen Computer
nennt man Nodes. Jede Node kommuniziert rund um die Uhr mit ihren benachbarten Nodes. Nodes können Informationen abfragen, 
weiterleiten und verifizieren. Das Bitcoin-Netzwerk ist demnach die Gesamtheit aller Nodes.

Was passiert aber, wenn eine Node für bösartige Zwecke falsche Informationen an andere Nodes verschickt? Dafür hat man ein
Vertrauens-System im Bitcoin-Netzwerk implementiert. Bevor eine Node empfangene Informationen abspeichert, prüft sie diese
auf Korrektheit in Bezug auf den derzeitigen Standard. Dieser Standard wird im BIP (Bitcoin Improvement Proposal) festgelegt,
welcher periodisch immer neue Richtlinien festlegt. () Natürlich hat das BIP keine Macht über die Nodes, diese können nämlich
selber entscheiden, ob sie nun nach den neuen Richtlinien arbeiten wollen oder nicht. Jedoch übernehmen die meisten Nodes mit
einem kurzen Update den neuen Standard, da dieser lediglich für das Interesse der Teilnehmer entwickelt wird.

Bitcoin Nodes sind kaum leistungshungrig und benötigen wenig Energie. Deswegen ist es für so gut wie jede Person möglich,
selber zu Hause eine Bitcoin Node zum Laufen zu bringen. Die Zwecke für diese Node könnnen aber von Person zu Person variieren.
Zum einen gibt es Leute, die mit ihrer Node nur mithören wollen, zum anderen aber auch diejenigen, die durch Mining Geld
verdienen wollen. Aus diesem Grunde gibt es verschiedene Implementierungen und Rollen von Nodes.

\subsection{Full Node}
Eine Full Node benötigt von allen Nodetypen am meisten Speicherplatz, ist dafür aber die mächtigste und fasst alle Funktionen in 
einer Node zusammen. Der distinkte Unterschied von der Full Node ist, dass diese die komplette Blockchain herunterlädt und
verifiziert. Der benötigte Speicherplatz der Blockchain liegt bei 423GB (Stand: 25/8/2022) und steigt pro Monat um etwa 3GB an.

\subsection{Mining Node}
Damit die Blockchain erweitert wird und alle Transaktionen verfiziert werden könnnen, müssen sogenannte Miner alle 10 Minuten
eine Lösung zu einem mathematischen Problem finden, indem sie mit der Hashfunktion SHA256 einen bestimmten Wert suchen. Das muss
vor allem schnell und präzise geschehen, damit die Chance, die Lösung als Erstes zu finden, maximiert wird. Mining Nodes 
profitieren vor allem von modernen GPUs, welche hunderte Millionen Hashes pro Sekunde berechnen.

\subsection{SPV Client}
Die wahrscheinlich beste Node fÍür den Otto Normalverbraucher ist der SPV Client (Simplified Payment Verification Client). Dieser
installiert lediglich eine Wallet, eine Kopie der Block Header und stellt eine Verbindung zum Bitcoin-Netzwerk her. Er ist einzig
und allein dafür ausgestattet, Transaktionen auf das eigene Konto zu empfangen und Geld zu versenden. 


\section{Die Blockchain}
Die Blockchain ist eine Datenstruktur, welche Transaktionen in miteinander verbundenen Blöcken speichert. Visualisieren kann man
diese wie einen Turm von Blöcken. Der unterste Block dient als Fundament und die Anzahl der Blöcke bestimmt die Höhe. Ein neuer
Block wird hinzugefügt, wenn ein Miner eine neue Lösung mit der SHA256 Hashfunktion gefunden hat. Der Miner kann dann aussuchen,
welche Transaktionen er in seinem Block inkludieren möchte. Bei einer Transaktion kann man eine beliebig hohe Spende an den Miner
abgeben. Transaktionen mit einer hohen Spende haben eine größere Wahrscheinlichkeit, im nächsten Block inkludiert zu werden.

Nun kann es jedoch passieren, dass zwei Miner gleichzeitig eine Lösung finden und im Bitcoin-Netzwerk zwei verschiedene Versionen
der Blockchain im Umlauf sind. Diese Diskrepanz wird aufgelöst, sobal einer der Ketten einen nächsten Block bekommt. Der Grund
dafür ist, dass das Bitcoin-Netzwerk nach einem Konzept namens "Proof of work" arbeitet. Wenn ein neuer Block hinzugefügt wird,
kann man sich sicher sein, dass ein Miner Prozesskraft und Energie dafür aufopfern musste. Die Kette, welche im Endeffekt mehr
Blöcke besitzt, wird als gültig anerkannt.

Einer der bekanntesten Wege, um das Bitcoin-Netzwerk anzugreifen, ist der "51\% Attack". Wenn man mit bösartigen Absichten
in Besitz von mehr als 50\% der Mining-Kraft kommt, kann man zwei schwerwiegende Dinge tun. Zum einen kann das Bitcoin-Netzwerk
komplett lahmgelegt werden, indem keine neuen Transaktionen in den Blöcken inkludiert werden. Zum anderen können Zahlungen
rückgängig gemacht werden, indem man in der einen Kette die Zahlung akzeptiert hat, dann jedoch eine andere Kette ohne diese 
Zahlung weiterführt und verlängert. Ein Angriff dieser Art wird jedoch immer unwahrsheinlicher, da die Warscheinlichkeit für
einen solchen Angriff exponentiell abfällt. (vgl Nakamoto, 8)

\subsection{Struktur eines Blocks}
Ein Block ist ein Behälter für eine Datenstruktur, welche für die Einschließung von Transaktionen in der Blockchain sorgt. Der
Block besteht aus einem Header, welcher für die Identifizierung des Blocks notwendig ist, gefolgt von einer Liste aller 
Transaktionen. Im Block Header befinden sich Informationen wie die derzeitige Schwierigkeit zur Lösung des Problems der Miner,
einen Zeitstempel, welcher die Entstehung des Blocks angibt und einer Zusatzzahl, welche nützlich für Miner ist.

Zur Identifizierung von Blöcken gibt es zwei Wege: Die Berechnung des Block Header Hashes und die Höhe des Blocks. Einen Block 
durch seine Höhe zu identifizieren scheint simpler, jedoch gibt es hierbei ein Problem. Bei dem vorher angesprochenen Fall,
dass zwei Miner gleichzeitig einen Block finden, scheitert die Identifizierung eines Blocks durch dessen Höhe. Um den Block
Header Hash zu berechnen, wird der Block Header (Schwierigkeit, Zeitstempel, Zusatzzahl) zwei mal mit dem SHA256 Algorithmus 
gehashed. Hierbei kommt ein 32-byte Hash heraus, welcher mit vielen Nullstellen beginnt. Jeder Block hat einen distinkten Block
Header Hash, wodurch es keine doppelte Blöcke geben kann. Der erste Block (Genesis-Block) der Blockchain hat einen Hashwert von
000000000019d6689c085ae165831e934ff763ae46a2a6c172b3f1b60a8ce26f.


\subsection{Genesis-Block}
Der Genesis-Block ist der erste Block der Blockchain mit einer Höhe von 0. Anders als die darauffolgenden Blöcke gilt dieser als
unveränderbar. Kreiert wurde dieser 2009 und wenn man jeden einzelnen Block auf der Blockchain zurückverfolgt, gelangt man
schlussendlich an den Genesis-Block. Standardmäßig fügt jede Implementierung von Bitcoin den Genesis-Block schon bei der 
Installation hinzu.