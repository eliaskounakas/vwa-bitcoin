\subsection{Bitcoin-Adressen generieren}
In Bitcoin werden Adressen verwendet, um Transaktionen zu empfangen und zu senden. Eine Bitcoin-Adresse ist eine Zeichenfolge, 
die aus einer Reihe von Zahlen und Buchstaben besteht und einem öffentlichen Schlüssel entspricht.

Um eine Bitcoin-Adresse zu generieren, wird der öffentliche Schlüssel mithilfe einer kryptographischen Hashfunktion verarbeitet.
Die Hashfunktion nimmt den öffentlichen Schlüssel als Eingabe und gibt einen Hash als Ausgabe. Der öffentliche Schlüssel selbst
wird nicht als Adresse genutzt, weil der Hash weniger Speicherplatz benötigt. Der Hash wird dann mithilfe einer weiteren 
Mathematikfunktion, der Base58Check-Codierung, in eine lesbare Adresse umgewandelt.

Um große Zahlen mit wenigen Zeichen darzustellen, nutzen Computersysteme häufig alphanumerische Zahlensysteme mit einer Basis,
die größer als 10 ist. Das Dezimalsystem nutzt die 10 Ziffern 0-9, während das Hexadezimalsystem zusätzlich die Buchstaben A-F
nutzt, wodurch es 16 Ziffern hat. Deshalb sind Zahlen, welche hexadezimal dargestellt werden, kürzer. Base64 ist ein weiteres
Zahlensystem, das noch viel kompakter ist. Es nutzt die 26 Buchstaben des Alphabets doppelt, nämlich in Klein- und Großschreibung.
Zusätzlich nutzt Base64 die 10 Ziffern 0-9 und die Symbole + sowie /.

Base58 hingegen ist eine Ableitung von Base64, welche ähnlich aussehende Symbole vermeidet, da sie in manchen Schriftarten
identisch aussehen können. Diese Symbole sind die Zahl null, der Buchstabe O (großes o), l (kleines L), I (großes i), das 
Zeichen + und /. Das vollständige Alphabet von Base58 ist:
\[ 123456789ABCDEFGHJKLMNPQRSTUVWX \]
\[ YZabcdefghijkmnopqrstuvwxyz \]

Base58Check wiederum ist ein Codierungsformat von Base58, welches oft in Bitcoin genutzt wird und einen eingebauten
Fehlerüberprüfungscode besitzt. Die Prüfsumme wird vom Hash der codierten Daten abgeleitet und kann deswegen genutzt werden, 
um Tippfehler und andere Fehler zu erkennen. Wenn die Prüfsumme also nicht mit den codierten Daten übereinstimmt, gilt die
eingegebene Bitcoin-Adresse als ungültig. Dies verhindert den Versand an falsche Bitcoin-Adressen und dadurch den Verlust
von Geld. \skpcf[67]{anto}

Die meisten Daten in Bitcoin werden Base58Check-codiert, was es einfacher macht, mit den Daten zu arbeiten. Zudem wird ein Präfix
bei jeder Base58Check-Codierung genutzt, welche den Datentyp angibt, um welchen es sich handelt. Das ermöglicht dem Nutzer, schnell
zu erkennen, welche Art von Daten dargestellt werden. Beispielsweise haben Bitcoin-Adressen in Base58Check den Präfix 0x00, welcher
umgewandelt auf Base58 1 ergibt. Private Schlüssel wiederum haben den Präfix 0x80 in Base58Check bzw. 5 in Base58.

\subsection{Private Schlüssel}
Im Bitcoin-Netzwerk ist der private Schlüssel eine geheime Zahl, die genutzt wird, um Bitcoin-Ausgaben ausgehend von einer 
bestimmten Bitcoin-Adresse zu autorisieren. Der private Schlüssel ist eine kritische Komponente der kryptographischen Sicherheit,
auf welche das Bitcoin-Netzwertk aufbaut. Genutzt wird der private Schlüssel, um Authentizität bei Transaktionen zu beweisen und
versichert dabei, dass die Transaktionen nicht verändert oder gefälscht sind.

Meist sind private Schlüssel in einer sogenannten Wallet, eine Applikation, welche dem Nutzer ermöglicht, Transaktionen zu tätigen,
gespeichert. Jede Wallet hat einen einzigartigen privaten Schlüssel, welche für Unterschriften bei Transaktionen genutzt wird und
den Besitz der jeweiligen Bitcoins beweist. Es ist essenziell, dass der private Schlüssel geschützt ist und mit niemanden geteilt
wird, da jeder mit Zugriff auf den privaten Schlüssel Bitcoins ausgeben kann.

Private Schlüssel werden als Abfolge von Zahlen und Buchstaben dargestellt. Genauer gesagt werden private Schlüssel durch ein
bestimmtes Format wie das Base64 Zahlensystem codiert. Folgende Hexadezimalzahl zeigt, wie ein privater Schlüssel im 
Normalfall aussieht: \[8a708d03d461a5c5839b53da4c40912ca094cc0edee8574a62d6895d033db7ea\] Diese Zahl entspricht einem Dezimalwert 
von ungefähr \( 6.2 * 10^{70} \). Generiert werden private Schlüssel mit verschiedenen Methoden wie beispielsweise ein 
Zufallszahlengenerator. Gespeichert werden private Schlüssel im Normalfall auf einem Computer oder Smartphone in einer sicheren 
Datei.

\subsubsection{Private Schlüssel generieren}
Prinzipiell sind private Schlüssel eine zufällige Zahl zwischen 1 und \( 2^{256} \) bzw. \( 1,15 * 10^{77} \). Um diese enorme Zahl
zu relativisieren: die geschätzte Anzahl der Atome im sichtbaren Universum beträgt \( 10^{80} \). \skpcf[59]{anto} Die Chance,
zweimal denselben privaten Schlüssel zu generieren, ist also extrem gering. Die einfachste Methode, um einen privaten Schlüssel
zu generieren, ist der Münzwurf: Kopf und Zahl wird jeweils 0 und 1 zugewiesen. Danach muss die Münze 256-mal geworfen werden
und man erhält eine 256-Bit lange Zahl. Prinzipiell kann man mit allem, was zufällig geschieht, einen privaten Schlüssel generieren.
Natürlich möchte man nicht für jeden privaten Schlüssel 256 Münzwürfe machen, deswegen werden hierfür Rechner genutzt. Ein großes
Problem in der Computerwelt ist es jedoch, zufällige Zahlen erzeugen zu können. Computer sind schlicht und ergreifend nicht dazu 
geeignet, Zufälle generieren zu können.

\subsubsection{Natürliche Zufallsquellen}
Aufgrund des Kerchoff'schen Prinzips muss die Definition eines Zufallszahlengenerators, der für Kryptografie geeignet ist, beinhalten, 
dass selbst wenn jedes Detail über den Generator bekannt ist (Schaltplan, Algorithmen usw.), er immer noch vollständig unvorhersehbare 
Bits erzeugen muss. Im Gegensatz zu PRNGs(Pseudorandom number generator) extrahieren physische (echte Hardware) Zufallszahlengeneratoren 
Zufälligkeit aus physischen Prozessen, die auf fundamental undeterministischer Weise verhalten, was sie zu besseren Kandidaten für die 
Erzeugung von wahren Zufallszahlen macht. Physische Prozesse sind \skpcf[5f]{mario}:
\begin{itemize}
    \item Geräusche (Radiowellen, Atmosphärische)
    \item harmonische Oszillatoren
    \item Quantenfluktuation
  \end{itemize}
Diese VWA beschäftigt sich mit den geräusch-basierten Zufallsgeneratoren.

Wärmerauschen ist eine Art von Rauschen, das in elektrischen Schaltungen auftritt und durch die thermische Bewegung von Ladungsträgern
(normalerweise Elektronen) in elektrischen Leitern verursacht wird. Die thermische Bewegung der Ladungsträger in einem elektrischen Leiter führt 
zu Schwankungen im Stromfluss, die als Rauschen interpretiert werden können. Dieses Rauschen ist proportional zur Temperatur des Leiters und der 
elektrischen Leitfähigkeit des Materials. Es ist auch unabhängig von der Art des Leiters und tritt in allen elektrischen Leitern auf, 
einschließlich Metallen, Halbleitern und sogar Isolatoren. Das Johnson-Nyquist-Rauschen ist ein weißes Rauschen, das bedeutet, dass die 
Frequenzverteilung des Rauschens gleichmäßig über einen breiten Frequenzbereich verteilt ist. Es hat auch eine bestimmte Spectral Density 
(Leistung pro Frequenzband) die proportional zur Temperatur und der elektrischen Leitfähigkeit des Leiters ist.

Eine weitere Art des Rauschens ist das chaotische Rauschen, das durch chaotische Systeme erzeugt wird. Chaotische Systeme sind Systeme, die sehr 
empfindlich auf kleine Änderungen in den Anfangsbedingungen reagieren und dazu neigen, unvorhersehbare und zufällige Muster zu erzeugen. Ein Beispiel 
für ein chaotisches System, das zur Erzeugung von Chaos-Noise verwendet werden kann, ist der Lorenz-Attraktor, die auf drei gekoppelten 
Differenzialgleichungen basiert und eine Art von chaotischem Verhalten erzeugt, das als Butterfly-Effekt bekannt ist. \skpcf[3ff]{jonck}

Manche Arten von Rauschen (insbesondere Johnson-Rauschen) erzeugen sehr kleine Spannungen, die vor der Umwandlung in digitale Form stark verstärkt
werden müssen. Die starke Verstärkung führt zu weiteren Abweichungen von der Zufälligkeit aufgrund der begrenzten Bandbreite und nicht-linearen 
Verstärkung des Verstärkers. Außerdem kann schnelles elektrisches Schalten von binärer Logik, die in der RNG-Schaltung verwendet wird, starke 
elektromagnetische Störungen verursachen, sodass mehrere RNGs in der Nähe (insbesondere auf einem Chip) tendenziell zur gegenseitigen 
Synchronisierung neigen, was zu einem starken Rückgang der Gesamtentropie führt. Darüber hinaus können empfindliche Verstärker durch externe 
elektromagnetische Felder leicht manipuliert werden, was für Kryptoangriffe ausgenutzt werden kann. \skpcf[7]{mario}


\subsection{Öffentliche Schlüssel}
Um Bitcoin-Adressen ihre irreversiblen Eigenschaften zu verleihen, wird das Public-Key-Kryptosystem bzw. das asymmetrische
Kryptosystem genutzt. Dieses System wurde 1970 basierend auf mathematischen Funktionen erfunden und ist Grundlage für die heutige
Cybersecurity.

Im Laufe der Geschichte wurden mehrere mathematische Funktionen entdeckt, die für Public-Key-Kryptosysteme verwendet werden können
wie beispielsweise das Potenzieren von Primzahlen und die Multiplikation eines bestimmten Punktes auf einer elliptischen Kurve
(diese Funktion wird auch von Bitcoin genutzt). Das besondere an diesen Funktionen ist, dass man sie nur in eine Richtung
berechnen kann bzw. ist es sehr viel schwerer, sie in die andere Richtung zu berechnen. Basierend auf diesen mathematischen
Funktionen erlaubt die Kryptographie es, digitale Geheimnisse und nicht fälschbare Unterschriften zu kreieren. \skpcf[62]{anto}

Zwischen dem privaten und öffentlichen Schlüssel gibt es eine mathematische Beziehung. Der öffentliche Schlüssel wird nämlich
aus dem privaten Schlüssel mit der Multiplikation eines Punktes auf einer elliptischen Kurve generiert. Aus dieser mathematischen
Beziehung können digitale Unterschriften generiert werden. Die digitale Unterschrift kann auf Korrektheit einzig und allein mit
dem öffentlichen Schlüssel geprüft werden. Um jedoch eine Unterschrift generieren zu können, benötigt man den privaten Schlüssel.
Das bedeutet, dass jeder prüfen kann, ob eine Transaktion valide ist, aber nur derjenige beziehungsweise diejenige mit Zugriff auf den 
privaten Schlüssel kann Transaktionen tätigen.

Bei einer vollständigen Transaktion muss der Besitzer beziehungsweise die Besitzerin sowohl den öffentlichen Schlüssel als auch die
dazugehörige Unterschrift, welche bei jeder Transaktion anders ist, vorzeigen, um Bitcoins ausgeben zu können. Daraufhin kann 
jeder Knotenpunkt im Bitcoin-Netzwerk für sich selbst entscheiden, ob die Transaktion korrekt ist und somit bestätigen, dass 
der Versender beziehungsweise die Versenderin tatsächlich zum Zeitpunkt der Transaktion die Bitcoins besitzt.

Einfach dargestellt ist die Berechnung des öffentlichen Schlüssels durch die Multiplikation elliptischer Kurven unter Verwendung
des privaten Schlüssels wie folgt:

\[ K = k * G \]

k ist der private Schlüssel, G ist eine Konstante mit dem Namen ''Generator Point''und K ist der resultierende öffentliche 
Schlüssel. In diese Richtung ist die Berechnung relativ simpel, in die andere jedoch fast unmöglich. Um den Wert von k 
herauszufinden, wenn man K kennt, bleibt nichts anderes übrig, als alle möglichen Werte für k einzusetzen, bis man auf den Wert 
von K stößt. \skpcf[65]{anto} Demnach nennt man Funktionen wie diese Einwegfunktionen.



