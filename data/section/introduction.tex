Bitcoin basiert auf einem dezentralen Netzwerk von Computern, die die Transaktionen validieren und auf einer öffentlich 
zugänglichen Blockchain aufzeichnen. Eine Blockchain ist im Grunde eine dezentrale Datenbank, die alle Transaktionen enthält, 
die jemals in einem Netzwerk durchgeführt wurden. Jede Transaktion wird von den Benutzern des Netzwerks validiert und durch ein 
komplexes kryptografisches Verfahren gesichert, um sicherzustellen, dass sie nicht manipuliert werden kann.

Das Ziel dieser Arbeit besteht darin, die innere Funktionsweise von Bitcoin und allen dazugehörigen Technologien, besonders die
Blockchain, zu untersuchen. Es wird untersucht, wie das Bitcoin-Netzwerk aufgebaut ist, wie es zustande kam und welche
Individuen daran teilnehmen können. Des Weiteren wird die Eröffnung eines Kontos im Bitcoin-Netzwerk erläutert und wieso diese Konten 
praktisch nicht kompromittierbar sind. Außerdem wird gezeigt, wie Transaktionen getätigt werden können und wie diese ablaufen.

Hierbei handelt es sich um eine reine Literaturarbeit. Das Hauptwerk, auf welches sich bezogen wird, ist ''Mastering Bitcoin''
von Andreas Antonopoulos. (Antonopoulos 2017) Es beschreibt vor allem Aufbau und Funktion des Bitcoin-Netwerkes und wie 
alle Einzelteile zusammen arbeiten. Für unterstützende Technologiem, welche Bitcoin benötigt, wird vor allem Mario Stipčevićs
Arbeit ''True Random Number Generators'' zur Entdeckung von natürlichen, zufälligen Zahlen genutzt. (Stipčević 2014) Außerdem
wurde eine Bitcoin-Full-Node von zu Hause aus zur Erforschung der Eigenschaften des Bitcoin-Netzwerkes betrieben.

\subsection{Was ist Bitcoin?}
Häufig wird Bitcoin einzig und allein mit der Kryptowährung selbst assoziiert. Der Begriff Bitcoin umfasst jedoch alle 
Konzepte und Technologien, die das digitale Zahlungssystem ermöglichen. \skpcf[1]{anto} Eines dieser Konzepte ist das Bitcoin-Netzwerk, wodurch
Nutzer Zahlungen propagieren und verifizieren können. Mit der Kryptowährung Bitcoin selbst kann man Käufe tätigen, Geld an 
Verwandte oder Organisationen versenden und Waren verkaufen, genauso wie es bei einer herkömmlichen Währung der Fall ist. Der 
große Unterschied zwischen Bitcoin und herkömmlicher Währung ist, dass es kein handgreifliches Bitcoin gibt. Die Menge an 
Bitcoin, die man besitzt, wird durch die Zusammenfassung aller erhaltenen Transaktionen auf die eigene Bitcoin-Addresse erlangt. 

Als Grundlage für Transaktionen in Bitcoin dient die asymmetrische Kryptographie, ein kryptographisches System, welches auf
einem Schlüsselpaar basiert. Das Schlüsselpaar setzt sich aus dem privaten Schlüssel und öffentlichen Schlüssel zusammen. Mit 
dem öffentlichen Schlüssel wird die Bitcoin-Addresse des Nutzers abgeleitet, welche benötigt wird, um Zahlungen zu empfangen. 
Ähnlich wie der PIN oder das Passwort bei einem herkömmlichen Bankkonto lassen sich Zahlungen mit dem privaten Schlüssel 
im Bitcoin-Netzwerk tätigen.

In Bitcoin gibt es keine zentrale Macht oder Bank, die neues Geld produziert. Stattdessen werden neue Bitcoins mit einem Prozess
namens Mining gewonnen, welcher die Lösung zu einem mathematischen Problem sucht und gleichzeitig neue Transaktionen verifiziert.
Alle 10 Minuten wird eine neue Lösung gefunden und alle Miner starten wieder von vorne. Das bedeutet, dass jeder Nutzer auch ein
Miner sein kann und Transaktionen maximal 10 Minuten brauchen, um verifiziert zu werden.

Die kleinste Einheit von Bitcoin ist der Satoshi, welcher nach dem Erfinder von Bitcoin, Satoshi Nakamoto benannt wurde. Ein 
Bitcoin entspricht dem Wert von 100 Millionen Satoshis. Da der Wert von Bitcoin in den letzten Jahren exponentiell gestiegen ist,
gewann der Satoshi immer mehr an Bedeutung. Transaktionen haben meist einen 8-stelligen Dezimalwert unter 1 (Bsp.: 0,00140209 BTC),
was konventionell schlecht darstellbar ist. Hier wäre es sinnvoll, den Wert in Satoshis anzugeben, also 140 209 Satoshis.


\subsection{Geschichte von Bitcoin}
2008 veröffentlichte jemand unter dem Alias Satoshi Nakamoto "Bitcoin: A Peer-to-Peer Electronic Cash System". \skpcf[1]{nakamoto}
Mit dieser Arbeit erfand Nakamoto die erste Kryptowährung Bitcoin. Er kombinierte frühere Erfindungen mit seinen Ideen und 
erfand somit das erste komplett dezentralisierte, elektronische Zahlungssystem. Die Schlüsselrolle spielte die Implementierung 
von einem verteilten Aufwand der Prozessorkraft, wodurch alle 10 Minuten demokratisch für die korrekte Folge von Aktionen 
abgestimmt werden kann.

In dem Jahr 2011, als Nakamoto sich von seinem Projekt abwandte und es nicht länger optimierte, stieg Bitcoin mit einem
Aufschwung von 3000\% in nur drei Monaten von 1USD auf 32USD. Explodiert ist der Wert jedoch erst 2018, als Bitcoin einen
Höchstwert von ~19,000USD erreichte. Im vierten Quartal des Jahres 2021 hat Bitcoin seinen bislang höchsten Wert (Stand: 24/08/2022)
von ~67,000USD erreicht.