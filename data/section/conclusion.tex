Nach dem Schreiben dieser Arbeit ergibt sich die Erkenntnis, dass Bitcoin ein unglaublich vielseitiges
Thema ist, wenn man über den rein finanziellen Aspekt hinwegblickt. Abgesehen von den etlichen Nutzern
der Währung gibt es auch wahnsinnig viele, aber notwendige Kleinigkeiten, welche Bitcoin überhaupt erst
ermöglichen. Diese Arbeit kratzt jedoch nur an der Oberfläche, besonders im Bereich der Kryptographie.
Die Mathematik hinter Bitcoin würde nämlich eine eigene wissenschaftliche Arbeit an sich benötigen, wenn
nicht sogar mehrere. 

Des Weiteren hat sich ergeben, dass das Teilhaben an Bitcoin gar nicht so kompliziert sein muss. Die zu Hause
aufgestellte Full Node lies sich überraschend einfach bedienen. Wenn man Bitcoin nur zum Kauf oder Verkauf
nutzen möchte, ist die Sache nochmal um einiges einfacher. Mehr als eine Wallet und etwas Guthaben benötigt
man hierfür nicht.

Arbeitstechnisch habe ich Fehler gemacht, die mir sehr viel Zeit kosteten. Bevor ich überhaupt anfing zu
schreiben, habe ich versucht, mir die ganze Literatur einzuprägen. Das hat logischerweise nicht gut
funktioniert und ich musste mir die Literatur beim Schreiben nochmals durchlesen. Zudem wollte ich beim
Schreiben stetig mehr Themen behandeln, welche ich ursprünglich gar nicht vorgeplant hatte. Bitcoin ist
möglicherweise auch einfach ein etwas zu breites Thema für eine VWA. Zukünftig werde ich versuchen, mein 
Thema gezielter einzugrenzen und bewusster zu schreiben.