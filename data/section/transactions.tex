Bitcoin-Transaktionen sind Aufzeichnungen über Übertragungen von Bitcoins von einem Benutzer an einen anderen. Jede
Bitcoin-Transaktion besteht aus mindestens einer Eingabe- und einer Ausgabeadresse. Die Eingabe-Adresse stellt die Bitcoins
bereit, die übertragen werden sollen, und die Ausgabe-Adressen empfangen die übertragenen Bitcoins. Eine Bitcoin-Transaktion
kann auch eine Gebühr beinhalten, die an die Miner gezahlt wird, die die Transaktion bestätigen und in die Blockchain aufnehmen.
Es ist wichtig, zu beachten, dass Bitcoin-Transaktionen irreversibel sind, d.h. sie können nicht rückgängig gemacht werden,
nachdem sie einmal bestätigt wurden. Daher sollte man sorgfältig darauf achten, an die richtige Adresse zu senden, bevor man eine
Bitcoin-Transaktion durchführt.

\subsection{Wallets}
Eine Wallet besteht aus Software, die private und öffentliche Schlüssel enthält und die Blockchain nutzt, um Währung zu senden
und zu empfangen. Die Währung in diesen Wallets wird in Form von Coins, wie beispielsweise Bitcoin, hinzugefügt. Um Coins oder
Währung empfangen oder senden zu können, muss eine Wallet erstellt werden. Die Währung selbst ist nicht an einem Ort in der Wallet
gespeichert, sondern befindet sich in der Form von Transaktionsaufzeichnungen in der Blockchain. \skpcf[4]{yadav} Demnach enthält 
die Wallet lediglich die Informationen (Schlüssel), um Währung auszugeben, nicht aber die Währungselbst.

Es gibt zwei primäre Arten von Wallets, welche sich darin unterscheiden, ob die enthaltenen Schlüssel zusammenhängend sind oder
nicht. Die erste Art einer Wallet ist die nichtdeterministische Wallet, bei welcher jeder Schlüssel unabhängig von einer 
zufälligen Zahl generiert wird. Die Schlüssel in dieser Wallet sind NICHT zusammenhängend. Die zweite Art einer Wallet ist die
deterministische Wallet, bei welcher alle Schlüssel Derivate von einem Masterschlüssel, welcher auch als Samen bezeichnet wird.
Alle Schlüssel in dieser Art von Wallet sind zusammenhängend und können neu generiert werden, solange man den originalen Samen
beziehungsweise den Masterschlüssel hat. Es gibt mehrere Methoden zur Derivation von Schlüsseln, welche in deterministischen
Wallets genutzt werden. Die am häufigsten genutzte Derivationsmethode nutzt eine baumähnliche Struktur und wird als 
hierarchische deterministische (HD) Wallet bezeichnet. \skpcf[93f]{anto}

\subsubsection{Deterministische Wallets}
Deterministische Wallets beinhalten private Schlüssel, welche alle von einem gemeinsamen Samen abstammen. Die privaten Schlüssel
werden vom Samen durch Einwegfunktionen beziehungsweise Hashfunktionken berechnet. Der Samen ist eine zufällig generierte Zahl,
welche mit anderen Daten wie zum Beispiel einer Index Nummer kombiniert wird, um dann mittels einer Hashfunktion einen privaten
Schlüssel generiert. Im Gegensatz zu der nichtdeterministischen Wallet benötigt man bei der deterministischen Wallet einzig und 
allein den Samen, um auf alle Schlüssel zugreifen zu können, da diese alle vom Samen abstammen. Aus diesem Grunde benötigt man
nur ein einziges Backup, welches bei der Erstellung der Wallet gespeichert wird. Vorteilhaft ist dabei auch die Mobilität der
Wallet: Man muss einzig und allein den Samen auf die neue Wallet-Umgebung transferieren, um wieder Zugriff auf alle Schlüssel
zu haben. \skpcf[95]{anto}

Im BIP-32 standard wurde die häufigste Art der deterministischen Wallets implementiert: die hierarchische deterministische Wallet.
HD Wallets enthalten Schlüssel, die in einer Baumstruktur angeordnet sind. Ein Schlüssel kann Töchterschlüssel generieren, von 
welchen wiederum Enkelschlüssel generiert werden können, und so weiter, bis in die Unendlichkeit. Abgesehen von der Mobilität
hat die HD Wallet noch zwei weitere Vorteile gegenüber nichtdeterministischen Wallets. Erstens können bestimmte Äste von
Schlüsseln genutzt werden, um ihnen organisatorische Bedeutung zu verleihen. Beispielsweise kann ein Ast von Schlüsseln genutzt
werden, um Zahlungen zu senden und ein andere Ast von Schlüsseln erhält dann das Wechselgeld. Zweitens können Nutzer von HD
Wallets eine Reihe von öffentlichen Schlüsseln generieren, ohne Zugriff auf die zugehörigen privaten Schlüssel zu haben. Dadurch 
können die Schlüssel auf unsicheren Servern genutzt. \skpcf[96]{anto}


\subsubsection{Nichtdeterministische Wallets}
Anfangs bestanden Bitcoin Wallets aus einer Sammlung von zufällig generierten privaten Schlüsseln. Der erste Bitcoin Client
hatte 100 Schlüssel vorprogrammiert und generierte bei Bedarf mehr, wobei jeder Schlüssel nur einmal genutzt wird. Dies mag
für die Sicherheit vor Hackern gut klingen, jedoch ist es wegen der fehlenden Verwandtschaft der Schlüssel sehr schwierig
zu verwalten. Keine der Schlüssel haben eine Verbindungen miteinander und es muss jeweils für jeden Schlüssel ein Backup
erstellt werden. Desto mehr Schlüssel generiert werden, desto aufwändiger ist es, alle Schlüssel zu verwalten und ein Backup
für diese zu erstellen.


\subsection{Ausgaben und Eingaben}
Die Blockchain ist ein global vereinbartes Register aller jemals durchgeführten Transaktionen und wird ständig als eine einzige 
verknüpfte Liste erweitert. Jede Transaktion in der Blockchain, zum Beispiel Alice, die Bob 10 Bitcoins zahlt, hat einen oder
mehrere Transaktionausgänge (TXO), die als Summen aus verwendbarer Bitcoins dienen. Hierbei muss beachtet werden, dass Bitcoin
Summen nicht in kleinere Summen aufgeteilt werden können, d.h., dass es im Normalfall einen Überschuss an Bitcoins gibt, der an 
den Sender zurückgezahlt wird, ähnlich wie beim Wechselgeld im Falle normaler Währung. Sobald Bob die TXO beziehungsweise 
die Summen seiner verwendbaren Bitcoins erhält, bezeichnet man sie als unausgegebene Transaktionausgänge (UTXO). Sie bleiben
UTXOs, bis der Besitzer (in diesem Beispiel Bob) sie einlöst, um jemand anderen zu bezahlen (in diesem Fall werden sie wieder
als ausgegebene TXOs bezeichnet). \skpcf[1]{pinto} Vor jeder Transaktion befindet sich eine vier-byte lange
Transaktionsversionnummer, welche anderen Peers und Minern sagt, mit welchen Regeln diese Transaktion bestätigt werden soll.
UTXOs haben eine Indexnummer basierend auf ihrer Lage in der Transaktion. Wenn ein beliebiger Bitcoin (oder nur ein Anteil)
zum ersten mal versendet wird, nachdem dieser gemined wurde, hat die zugehörige Transaktion eine Indexnummer von 0. Im Gegensatz
zu Ausgaben (UTXOs), haben Eingaben (TXOs) eine Ausgabenindexnummer, um den UTXO zu spezifisieren, der ausgegeben werden soll.
Zudem hat es einen Skript mit bestimmten Parametern, welcher erfüllt werden muss, um die Bitcoins ausgeben zu können.

Meist enthält eine Transaktion auch eine Transaktionsgebühr, welche an Miner als Ausgleich zur Versicherung des Bitcoin-Netzwerkes
ausgezahlt wird. Zudem schützen Transaktionsgebühren auch vor Hackerangriffen, indem sie es finanziell unmöglich machen, das 
Bitcoin-Netzwerk mit Transaktionen zu überschütten. Zudem können Miner entscheiden, welche Transaktion sie in ihrem Block
inkludieren möchten, d.h. eine Transaktion mit einer höheren Transaktionsgebühr wird eher inkludiert, als eine Transaktion mit
niedrigen Transaktionsgebühren. Berechnet werden Transaktionsgebühren anhand der Größe einer Transaktion in Kilobyte und nicht 
anhand der Menge von Bitcoins, die versendet werden. Im Laufe der Zeit änderte sich die Berechnung von Transaktionsgebühren
und die Bedeutung von Transaktionsgebühren bei der Prozessierungspriorität. Zuerst hatten Transaktionsgebühren einen festen 
Wert und waren konstant. Die Gebührenregeln lockerten sich jedoch immer mehr und erlaubten es, dass Marktkräfte den
Gebührenpreis durch Kapazität des Bitcoin-Netzwerkes und Transaktionsvolumen bestimmten. 2016 explodierte Bitcoin an Popularität,
wodurch das Kapazitätslimit erreicht wurde und die Transaktionsgebührenpreise in die Höhe schossen; Transaktionen ohne
Transaktionsgebühren sind dadurch Geschichte gewesen. \skpcf[127]{anto}

\subsection{Transaktionenskripts}
Bitcoin nutzt eine eigene Stapel-basierte Skriptsprache mit einer Reihe von Opcodes. Diese Skriptsprache, welche auch ganz 
einfach ''Script'' genannt wird, wird genutzt, um festzustellen, ob ein unausgegebener Transaktionsausgang (UTXO) ausgegeben 
werden kann, oder nicht. Wenn Bitcoin versendet wird,
dann wird lediglich ein Bitcoinwert zu einem Skript zugewiesen, welcher nur ausgegeben kann, wenn der Skript erfolgreich
ausgeführt wird und ein einziger ''true'' Wert auf dem Stapel entsteht. Den Skript, welcher dem UTXO zugewiesen ist, nennt man
''scriptPubKey''. Um die Bedingungen des Skriptes zu erfüllen, muss man eine korrekte ''scriptSig'' zur Verfügung stellen,
welche auch ein Skript ist und ausgeführt wird, um zu testen, ob der Versand von Bitcoin funktioniert hat. \skpcf[1ff]{jordan} 
Skript ist eine sehr einfach Sprache, welche dazu gemacht wurde, nicht viele Möglichkeiten zu bieten und auf fast jedem
Gerät ausführbar ist, wie zum Beispiel einem eingebetteten System. Die Sprache benötigt minimale Prozessorkraft und kann nicht 
viele der schicken Dinge tun, die moderne Programmiersprachen können. Zweck ist, dass eine Sprache, die nicht viele Möglichkeiten
bietet, sich einfach überprüfen lässt und sehr robust gegen Bugs o.Ä. ist, was sie zu einem perfekten Kandidaten für die 
Versendung von Währung in Bezug auf Sicherheit macht.

Bitcoins Skriptsprache wird Stapel-basiert genannt, weil sie eine Datenstruktur names ''Stapel'' nutzt. Ein Stapel ist ein sehr
einfaches Datenkonstrukt, welches man sich wie einen Stapel von Karten vorstellen kann. Ein Stapel hat zwei mögliche Operationen:
Push (drauflegen) und Pop (herunternehmen). Push legt ein Objekt an die Spitze des Stapels. Pop entfernt ein Objekt von der
Spitze des Stapels. Operationen auf einem Stapel können nur mit dem obersten Objekt des Stapels interagieren. Deshalb nennt man
dieses Konzept auch Last-In-First-Out-Prinzip (LIFO). Die Skriptsprache führt den Skript aus, indem sie jedes Objekt von link
nach rechts ausliest. Dabei werden Zahlen (Datenkonstanten) auf den Stapel hinaufgelegt. Operatoren können mehrere Elemente auf
einmal von dem Stapel herunternehmen und drauflegen, mit diesen interagieren und möglicherweise ein daraus resultierendes Objekt
zurück auf den Stapel legen. Beispielsweise kann OP\_ADD zwei Objekte von dem Stapel herunternehmen, diese addieren, und das
resultierende Ergebnis wieder auf den Stapel legen.
Bedingte Operatoren prüfen eine Bedingung und geben boolean Wert von TRUE oder FALSE zurück. Beispielsweise kann OP\_EQUAL zwei
Objekte vom Stapel herunternehmen und legt TRUE (wird in Script als 1 dargestellt) zurück hinauf, wenn die Elemente denselben
Wert haben und FALSE (wird in Script als 0 dargestellt), wenn die Elemente nicht denselben Wert haben. Im Normalfall haben
Bitcoin-Transaktions-Skripts einen bedingten Operator, um die Gültigkeit der Transaktion als gültig beziehungsweise TRUE oder
ungültig beziehungsweise FALSE zu bestimmen. \skpcf[133]{anto}

Im ursprünglichen Bitcoin Client (heute Bitcoin Core) waren der ''scriptPubKey'' beziehungsweise der Ausgabeskript und der 
''scriptSig'' beziehungsweise der Eingabeskript miteinander verkettet und in einer Sequenz ausgeführt worden. Aus 
Sicherheitsgründen wurde dies 2010 geänder, da es eine Verwundbarkeit gab, wodurch missgebildete Ausgabeskripts in der 
Lage waren, Daten auf den Stapel des Eingabeskripts zu legen und diesen dadurch zu korrumpieren. Heutzutage werden die Skripte
deswegen seperat ausgeführt und die Stapel werden zwischen den Ausführungen transferiert, wie folgend beschrieben: Zuerst wird
der Ausgabeskript mit dem Script Engine ausgeführt. Falls der Ausgabeskript ohne Fehler (d.h. keine verbleibenden Operatorn)
ausgeführt wird, wird der Hauptstapel kopiert und der Eingabeskript wird ausgeführt. Wenn das Ergebnis der Ausführung des 
Eingabeskriptes mit den Stapeldaten des kopierten Ausgabeskriptes übereinstimmt (d.h. das Ergebnis ist TRUE oder 1), hat der 
Ausgabeskript es geschafft, die Bedingungen des Eingabeskriptes zu erfüllen und ist daher erlaubt, die enthaltenen UTXOs
auszugeben. Resultiert irgendein anderes Ergebnis außer TRUE beziehungsweise 1, nachdem der kombinierte Skript ausgeführt wurde,
ist die Eingabe ungültig, weil er es nicht geschafft hat, die Bedingungen zu erfüllen, welche auf dem UTXO gesetzt sind.